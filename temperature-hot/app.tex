// Module 37-5



// এসাইনমেন্ট ৬ সাবমিট করে এই বোনাস মডিউল দেখতে শুরু করো। 

// Intermediate জাভাস্ক্রিপ্ট বোনাস ( রিলিজ হচ্ছে মডিউল ৩৭.৫)

// #module_release #intermediate_js_milestone #module_37_5

// জাস্ট ৫টা ভিডিও,বোনাস ভিডিও। মূলত আরেকটা উদারহণ দিয়ে API কল করার সিস্টেমটা ব্যাখ্যা করা হয়েছে। সেই ক্ষেত্রে এইবার ইউজ করা হয়েছে হট হট কিছু একটা। 



// .

// মাইলস্টোন ৬ টেকএওয়ে  

// এই মাইলস্টোন থেকে তুমি যদি আটটা জিনিস শিখতে চাও তাহলে নিচের এই আটটি জিনিস আরেকবার ভালো করে দেখে নাও- 

// ১. fetch বা async await ইউজ করে API থেকে কিভাবে ডাটা লোড করতে হয়। ডাটা অনেক সময় অনেকভাবে থাকে। সেই ডাটা কোনটা কখন array কখন অবজেক্ট এর ভিতরে আছে। সেটা বুঝে সেই অনুসারে ডাটা দেখানোর সিস্টেম 

// ২. arrow ফাংশন কিভাবে ইউজ করা হয়

// ৩. template string এ ডাইনামিকভাবে কিভাবে ডাটা যোগ করতে হয় 

// ৪. map, forEach, filter, find এইগুলা কখন কোনটা ব্যবহার করতে হয়, এদের মধ্যে পার্থক্য কি 

// ৫. let, const, var এদের মধ্যে ডিফারেন্স কি, কোন কোনটা ইউজ করতে হয়। 

// ৬. কোনটা দিয়ে array এর মধ্যে লুপ করতে হয়, কোনটা দিয়ে অবজেক্ট এর মধ্যে লুপ করতে হয়  

// ৭. spread কিভাবে ইউজ করা হয়, স্প্রেড অপারেটর দিয়ে কিভাবে array কপি করে ফেলে।  

// ৮. ES6 এর মধ্যে কিভাবে অবজেক্ট বা array এর destructure করে সেটা থেকে ভেরিয়েবল ডিক্লেয়ার করতে হয়। 



// .



// এই মাইলস্টোন থেকে তুমি যদি আরো দশটা জিনিস এ খেয়াল রাখতে চাও তাহলে সেগুলো হবে- 

// ১. JSON কি জিনিস, এইটা দিয়ে কি করে ?

// ২. == আর === এর ডিফারেন্স কি ? 

// ৩. Block scope এবং Hoisting সম্পর্কে হালকা হলেও ধারণা 

// ৪. closure স্কোপ সম্পর্কে ধারণা 

// ৫. Truthy আর Falsy সম্পর্কে হালকা একটু ধারণা 

// ৬. কোন কোন ক্ষেত্রে undefined হয় এবং null আর undefined এর মধ্যে পার্থক্য কি ? 

// ৭. GET আর POST এর মধ্যে পার্থক্য কি ?

// ৮. class আর অবজেক্ট কি জিনিস 

// ৯. bind, call, apply এর পার্থক্য কি ?

// ১০.জাভাস্ক্রিপ্ট এর this সম্পর্কে ধারণা

// .

// মাইলস্টোন ৭

// আগামীকাল থেকে শুরু হবে আরেকটা ইন্টারেস্টিং একটা মাইলস্টোন সেখানে কাজ হবে ব্রাউজার এর খুঁটিনাটি রিলেটেড কিছু জিনিস নিয়ে । ডিবাগ রিলেটেড কিছু কাজ কর্ম। আর এই মাইলস্টোন দিয়ে শেষ হবে তোমার কোর্সের অর্ধেক জার্নি। কি মজা ! নেক্সট মাইলস্টোন এর এসাইনমেন্ট এর পরে দুই দিনের রিভিশন/ক্যাচআপ।





// দেখা হবে নেক্সট মাইলস্টোন এ। 



